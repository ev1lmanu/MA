
\documentclass[pdftex,a4paper,11pt,DIV15,BCOR20mm,parskip,numbers=noenddot]{scrbook}

% \usepackage{default}
\usepackage[ngerman]{babel} % Deutsche Formattierung
\usepackage[T1]{fontenc}
\usepackage[utf8]{inputenc} % Zeichenkodierung
\usepackage[pdftex]{color} % Farben
\usepackage{setspace} % Das Paket setspace ermöglicht ein einfaches Umstellen von normalem, anderthalbfachen oder doppeltem Zeilenabstand.
\usepackage[pdftex]{graphicx} % Bilder
\usepackage[final]{listings}  % ermöglicht die Darstellung von (Programm-)Quellcode in der Arbeit
\usepackage[normalem]{ulem} % ermöglicht das Unterstreichen von Text
\usepackage{amsfonts} % ok-zeichen usw
\usepackage{amsmath} % mathekrams
\usepackage{scrpage2}
\usepackage{babelbib} % Bibliographie
\usepackage{array} % für tabellen
\usepackage[style=base,margin=10pt,font=footnotesize,labelfont=bf]{caption} % Formattierung Bildunterschrift
\usepackage{listings} % Quellcode
\usepackage{float} % floating figures
\usepackage{tikz} % Diagramme
\usepackage{wrapfig} % Text neben Abbildungen
\usepackage{pdflscape} % Seiten in Querformat
\usepackage{geometry}
\usepackage{soul}
\usepackage{textcomp} % Quote Symbol

% schrift auf palatino umstellen
\usepackage{palatino}
\setkomafont{sectioning}{\normalcolor\bfseries} 
% \renewcommand{\familydefault}{\sfdefault}

\definecolor{uhhred}{cmyk}{0,1,1,0}
\definecolor{beige}{RGB}{194,116,31}
\definecolor{green}{RGB}{41,128,38}
\definecolor{brown}{RGB}{132,60,36}
\definecolor{darkblue}{RGB}{38,38,128}

\newcommand{\routput}{\vspace{-0.25cm}\hspace{5pt}\small\color{blue}\ttfamily} 
\newcommand{\rsymbol}{\vspace{-0.25cm}\hspace{5pt}\small\color{blue}\ttfamily\textquotesingle}
\newcommand{\rerror}{\vspace{-0.25cm}\hspace{5pt}\small\color{red}\ttfamily$\bigotimes$ \hspace{0cm}}

\newcommand{\q}{\textquotesingle}
\newcommand{\qq}{\textquotedbl}
    
% Listing-Formattierung
\lstset{
    backgroundcolor=\color{white}, 		% white background
    %     columns=fullflexible, 			% allow latex to break lines
    numbers=none, 				% no line numbering
    showstringspaces=false, 			% no gap character in strings
    belowskip=-10pt, 				% remove blank space at the bottom of listing
    basicstyle=\ttfamily\small\color{darkblue}, 
    xleftmargin=5pt, % Padding
    language=Lisp, 				% for highliting comments and literals
    commentstyle=\color{beige}, 		% comment style
    stringstyle=\color{green}, 			% literal style
    numberstyle=\color{green},
    alsoletter={\#, >},
    keywordstyle=\ttfamily, 			% keyword style
    keywordstyle=[2]\color{green},		% style for literals
    keywordstyle=[3]\color{black},	% style for non-terminals
    literate=
	  *{(}{{{\color{brown}{(}}}}{1} 	% colored brackets
	  {)}{{{\color{brown}{)}}}}{1}	 	
	  {{[}}{{{\color{brown}{{[}}}}}{1}
	  {{]}}{{{\color{brown}{{]}}}}}{1}
	  {\{}{{{\color{brown}{\{}}}}{1}
	  {\}}{{{\color{brown}{\}}}}}{1}
	  {'}{{{\bfseries\color{green}{\textquotesingle}}}}{1}
	  {\#'}{{{\color{green}{\#\bfseries\textquotesingle}}}}{1},
%     % lists of keywords
%     morekeywords={},
    keywords=[2]{\#t, \#f},
    keywords=[3]{\#lang swindle, \#lang racket, >} 
}

\makeatletter
\newcommand{\thickhline}{%
    \noalign {\ifnum 0=`}\fi \hrule height 1.2pt
    \futurelet \reserved@a \@xhline
}
\newcolumntype{'}{@{\hskip\tabcolsep\vrule width 1.2pt\hskip\tabcolsep}}
\makeatother
  
%   % Tikz styles
%   \usetikzlibrary{shapes,arrows,decorations.markings}
%   \tikzstyle{block} = [rectangle, draw, fill=gray!15, text width=11em, text centered, rounded corners, minimum height=3em]
%   \tikzstyle{label} = [text centered]
%   \tikzstyle{arrow} = [draw, -latex', thick, postaction={decorate}] 
%   \tikzstyle{line} = [draw, -, thick]
  
\usepackage{etoolbox}
\makeatletter
\patchcmd{\chapter}{\if@openright\cleardoublepage\else\clearpage\fi}{}{}{}
\makeatother  
	
\usepackage{colortbl} % für farbige Tabellen
\usepackage{longtable} % für mehrseitige Tabellen
\renewcommand{\arraystretch}{1.25} % in Tabellen: Padding des Textes nach oben und unten in Prozent

% Fußnoten werden im gesamten Dokument fortlaufend hochgezählt und nicht nur kapitelweise, vgl. http://www.golatex.de/nummerierung-der-fussnoten-durchgehend-im-gesamten-dokument-t2042.html
\usepackage{chngcntr}
\counterwithout{footnote}{chapter}

\setlength{\emergencystretch}{1em} % Für den Fall, dass Zeilen im 1. Anlauf nicht richtig umgebrochen werden können, einen 'Notfallraum' einrichten (vgl. http://www.golatex.de/overfull-boxes-in-latex-t1979.html) 

% entnommen aus http://www.siart.de/typografie/latextipps.xhtml#floats
\renewcommand{\floatpagefraction}{0.8} % gibt den Bruchteil einer Seite, die für Gleitobjekte benutzt wird, an, der erreicht werden muss, bevor eine neue Seite angefangen wird. (Standard: 0.5; d.h. wenn ein Bild 51% der Seite einnimmt, wird extra für dieses Bild eine ganze Seite reserviert --> unschön)
\renewcommand{\topfraction}      {0.8}
\renewcommand{\bottomfraction}   {0.5} % \topfraction / \bottomfraction, gibt den Bruchteil einer Seite an, bis zu dem Gleitobjekte oben bzw. unten angeordnet werden sollen.
\renewcommand{\textfraction}     {0.15} % gibt den Bruchteil einer Seite an, der mit Text belegt werden können muss.
\makeatletter
  \setlength{\@fptop}{0pt} % Wenn ein Float-Objekt allein auf einer Seite steht, soll es am oberen Rand der Seite erscheinen und nicht vertikal zentriert
\makeatother

  \usepackage[
%   	pdfstartview={Fit},   
%   	pdffitwindow=true,
  	colorlinks,
  	linkcolor=black,
  	anchorcolor=black,
  	citecolor=black,
  	urlcolor=black,
  	bookmarks, 
  ]{hyperref}
   
% Zeilenabstand
% \setstretch{1.24}   

% Strafpunkte, die beim Seitenumbruch vergeben werden, falls die erste Zeile eines Absatzes allein auf der vorangehenden Seite verbleibt. vgl http://www.jr-x.de/publikationen/latex/tipps/zeilenumbruch.html
\clubpenalty=150

% Strafpunkte, die beim Seitenumbruch vergeben werden, falls die letzte Zeile gerade noch auf die nächste Seite umgebrochen wird. vgl http://www.jr-x.de/publikationen/latex/tipps/zeilenumbruch.html
\widowpenalty=150  

% Pagestyle definieren (nach Martins Template)
\defpagestyle{diplHeadings}
{ % es folgt: Definition des Seitenkopfes: 
  % obere Linie
	(0pt,0pt)
	% linke Seite
	{\upshape \rlap{\pagemark} \hfill \headmark \hfill} % auf einer linken Seite soll LINKS die Seitenzahl stehen und mittig die Headline (headmark)
	% rechte Seite
	{\upshape \hfill \headmark \hfill \llap{\pagemark}} % auf einer rechten Seite soll RECHTS die Seitenzahl stehen und mittig die Headline (headmark)
	% falls Layout "one page"
	{}
	% untere Line
	(\textwidth,1pt)
}
{ % es folgt: Definition des Seitenfußes: Wir wollen lediglich eine schwarze, über die komplette Seite gehende Linie erzeugen
  % obere Linie
	(\textwidth,1pt)
	% linke Seite
	{}
	% rechte Seite
	{}
	% falls Layout "one page"
	{}
	% untere Linie
	(0pt,0pt)
}  
% Pagestyle auch für Chapter-Anfang einrichten
\renewcommand*{\chapterpagestyle}{diplHeadings}
\renewcommand*{\chapterheadstartvskip}{\vspace*{-\topskip}}
\automark[section]{chapter}


% Ränder	
\setlength{\textwidth}{15cm}        % Textbreite
\setlength{\textheight}{24cm}       % Texthöhe
\setlength{\topmargin}{-12mm}       % oberer Rand