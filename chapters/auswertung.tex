\chapter{Auswertung}
Die Funktionsweise des Prototypen wird zunächst anhand des Pokemonbeispiels aus der Racketeinführung erprobt. Anschließend werden Einschränkungen und Grenzen des Prototyps aufgezeigt und in einem Fazit dargelegt, wie gut sich das Metaobjektprotokoll für die Erweiterung einer Sprache eignet.

\section{Test des Prototyps}
Das im Rahmen dieser Arbeit entstandene Modul kann mit \texttt{require} eingebunden werden, wenn es sich im gleichen Verzeichnis oder in \texttt{racket/collects} befindet. Für diesen Test befindet es sich im gleichen Verzeichnis.

\begin{lstlisting}
(require "multiple-inheritance.rkt")
\end{lstlisting}

In dem Pokemon-Beispiel aus der Racketeinführung gab es die drei Klassen Thing, Element und Animal. Das Problem, das sich in Racket zuvor nicht befriedigend lösen ließ, war eine Klasse Pokemon mit einer Methode \texttt{attack} zu definieren, die eine Kombination der Methoden der Superklassen ist. 

Durch Verwenden des in dieser Arbeit entstandenen Moduls lassen sich Mehrfachvererbung und Methodenkombination deutlich einfacher umsetzen. Mit \texttt{define/generic} ist es möglich, \texttt{attack} in der Klasse Thing als generische Funktion deklarieren:

\begin{lstlisting}
(define Thing (class object% (super-new)
                (init-field [name "a Thing"])
                (define/public (who-are-you?)
                  (string-append "I am " name "!"))
                ; generic function
                (define/generic (attack)
                  (compose reverse list))))
\end{lstlisting}

Das Ziel war eine Listenkombination. Das Ergebnis sieht am Ende jedoch mit einer umgekehrten Liste besser aus. Die Implementation der Methodenkombination erlaubt es praktischerweise, eine beliebige Funktion für die Kombination anzugeben, also auch eine Komposition von Funktionen.

Die Klasse verhält sich sonst genauso wie zuvor:

\begin{lstlisting}
> (send (new Thing) who-are-you?)
\end{lstlisting}
{\routput {\qq}I am a Thing!\qq}

\begin{lstlisting}
> (send (new Thing [name "Bob"]) who-are-you?)
\end{lstlisting}
{\routput {\qq}I am Bob!\qq}

In den Klassen Element und Animal lässt sich für \texttt{attack} eine implementierende Methode angeben:

\begin{lstlisting}
(define Element (class Thing 
                  (init [name "an Element"])
                  (super-new [name name])
                  (init-field [attr 'water])
                  (define/public (hot?) (equal? attr 'fire))
                  (define/override (who-are-you?)
                    (string-append (super who-are-you?)
                                   (if (hot?) " And I am hot!" "")))
                  ; implementing method
                  (define/public (attack) attr)))
                  
(define Animal (class Thing
                 (init [name "an Animal"])
                 (super-new [name name])
                 (init-field [size 'small])
                 ; implementing method
                 (define/public (attack) size)))
\end{lstlisting}

Auch Element und Animal verhalten sich sonst genauso wie zuvor.

\begin{lstlisting}
> (send (new Element) who-are-you?)
\end{lstlisting}
{\routput {\qq}I am an Element!\qq}

\begin{lstlisting}
> (send (new Element [attr 'fire]) who-are-you?)
\end{lstlisting}
{\routput {\qq}I am an Element! And I am hot!\qq}

\begin{lstlisting}
> (send (new Animal [name "Bob"]) who-are-you?)
\end{lstlisting}
{\routput {\qq}I am Bob!\qq}

Zusätzlich bieten nun beiden Klassen die neu definierte Funktion \texttt{attack}. Durch die Methodenkombination ist der Rückgabewert eine Liste.

\begin{lstlisting}
> (send (new Element) attack)
\end{lstlisting}
{\rsymbol (water)}

\begin{lstlisting}
> (send (new Animal) attack)
\end{lstlisting}
{\rsymbol (small)}

An Thing lässt sich die Funktion nicht aufrufen, denn es gibt keine implementierende Methode.

\begin{lstlisting}
> (send (new Thing) attack)
\end{lstlisting}
{\rerror No applicable methods found for generic function: attack}

Eine Definition der Klasse Pokemon ist einfach durch die Angabe von zwei Superklassen möglich. Auch Pokemon kann das geerbte Feld redefinieren und eine implementierende Methode für \texttt{attack} anbieten. Da die Felder \texttt{attr} und \texttt{size} für die Methodenkombination benötigt werden, ist es notwendig sie mit \texttt{inherit-field} sichtbar zu machen.

\begin{lstlisting}
(define Pokemon (class (Element Animal)  ; two superclasses
                  (init [name "a Pokemon"])
                  (super-new [name name])
                  ; implementing method
                  (define/public (attack) 'ball)
                  ; inherit fields that are needed for combination
                  (inherit-field attr size)))
\end{lstlisting}

Objekte der Klasse Pokemon haben automatisch alle Felder und Methoden der beiden Superklassen:

\begin{lstlisting}
> (send (new Pokemon) who-are-you?)
\end{lstlisting}
{\routput {\qq}I am a Pokemon!\qq}

\begin{lstlisting}
(define p (new Pokemon [name "Charmander"] [attr 'fire] [size 'big]))
> (send p who-are-you?)
\end{lstlisting}
{\routput {\qq}I am Charmander!\qq}

\begin{lstlisting}
> (send p  hot?)
\end{lstlisting}
{\routput \#t}

\begin{lstlisting}
> (get-field size p)
\end{lstlisting}
{\rsymbol big}

Die überschreibende Methode aus Element wird \emph{nicht} vererbt, da ausschließlich mit \texttt{define/public} definierte Methoden bei Mehrfachvererbung berücksichtigt werden. Pokemon zeigt daher das Standardverhalten aus der Klasse Thing. Die Klasse könnte die Methode natürlich selbst überschreiben.

Schließlich ist anhnand eines Aufrufs der \texttt{attack}-Funktion das Ergebnis der Methodenkombination sichtbar:

\begin{lstlisting}
> (send (new Pokemon) attack)
\end{lstlisting}
{\rsymbol (small water ball)}

\begin{lstlisting}
> (send p attack)
\end{lstlisting}
{\rsymbol (big fire ball)}

Methoden, für die eine generische Funktion existiert, können beliebig in der Vererbungshierarchie mit \texttt{define/public} redefiniert werden, ohne dass es einen Fehler gibt. Auch Thing kann eine implementierende Methode hinzufügen

\begin{lstlisting}
(define Thing (class object% (super-new)
                ...
                (define/public (attack) 'a)))
\end{lstlisting}

und es hat keinen negativen Einfluss auf die Klassen Element, Animal oder Pokemon. Der Wert wird einfach in die Methodenkombination mit aufgenommen.

\begin{lstlisting}
> (send (new Pokemon) attack)
\end{lstlisting}
{\rsymbol (a small water ball)}

Das Modul bietet damit einen deutlich einfacheren und intuitiveren Umgang mit Mehrfachvererbung als Mixins und Traits, wenngleich es einige Einschränkungen gibt.

\section{Einschränkungen und Grenzen}
Das Modul ist nur ein Prototyp. Das bedeutet, dass es  Einschränkungen und noch nicht implementierte Funktionalität gibt, die hier kurz angesprochen werden sollen.

Die Implementation beinhaltet noch keine Ergänzungsmethoden. Racket bietet bereits eine eigene Sammlung von Methoden, die geerbtes Verhalten überschreiben und ergänzen können. Das Verhalten der redefinierenden Methode lässt sich durch die Klassenoption \texttt{define/override} bereits erreichen. Vor- und Nachmethoden sind in Racket jedoch bisher nicht möglich und werden auch durch den Prototypen nicht abgedeckt.

Bei Einfachvererbung ohne Methodenkombination verhalten sich die Klassen genauso wie in Racket. Ansonsten ergeben sich Einschränkungen in Bezug auf die erstellte Superklasse und Methodenkombination:

Die Superklasse:
\begin{itemize}
 \item Es werden nur die Klassenoptionen \texttt{field}, \texttt{init-field} und \texttt{define/public} übernommen.
%  \item Felder befinden sich vor den Methoden in der Definition, unabhängig von der Reihenfolge in den ursprünglichen Klassen.
%  \item Die Reihenfolge von Methoden entspricht der Reihenfolge, in der sie geerbt werden.
 \item Die Superklasse enthält keine Methoden, zu denen es eine generische Funktion gibt.
 \item Die direkte Superklasse der neuen Superklasse ist \texttt{object\%}.
\end{itemize}

Generische Methoden und Methodenkombination:
\begin{itemize}
 \item Methoden, für die eine generische Funktion existiert, können im Gegensatz zu Racket beliebig mit \texttt{define/public} redefiniert werden.
 \item Felder von Superklassen, die für die Kombination benötigt werden, müssen mit \texttt{inherit-field} in der Klasse sichtbar gemacht werden.
 \item Es können nicht zwei generische Funktionen mit dem gleichen Namen definiert werden, auch nicht mit unterschiedlichen Parametern oder in voneinander unabhängigen Vererbungshierarchien.
 \item Die Kombinationsmethoden werden nach allen anderen Klassenoptionen in die Klassendefinition eingefügt und damit insbesondere nach dem \texttt{super-new}-Aufruf.
 \item Methodenkombination ist wie in CLOS nicht kompatibel mit ergänzenden Methoden wie \texttt{override} oder \texttt{augment}. Methoden, die nicht mit \texttt{define/public} definiert wurden, werden bei der Vererbung ignoriert.
 \item Da generische Funktionen innerhalb von Klassen definiert werden, ist es im Gegensatz zu CLOS nicht möglich, sie auf mehr als eine Klasse zu spezialisieren.
\end{itemize}

Diese Einschränkungen können dazu führen, dass einige der Klassenoptionen, die Racket bietet, nicht auf Klassen anwendbar sind, die durch Mehrfachvererbung entstanden sind oder für die eine generische Funktion definiert ist.

Der Prototyp eigent sich bereits für die Verwendung in der Lehre, wenn anstelle der Ergänzungsmethoden von CLOS die überschreibenden und ergänzenden Methoden von Racket behandelt werden. Auch in CLOS werden nur primäre Methoden kombiniert und in der Lehre verwendete Übungsaufgaben sind daran angepasst. Daher ist eine Einschränkung auf mit \texttt{define/public} definierte Methoden für die Lehre nicht problematisch. Die Angabe der Kombinationsart und die Definition eigener Kombinationen ist in dem Prototypen sogar deutlich einfacher als in CLOS.

Darüber hinaus lässt sich der Prototyp gut in Projekten nutzen, in denen Felder und Methoden überwiegend mit den unterstützten Schlüsselwörtern definiert werden. 

Vererbungshierarchien, in denen Methoden oft überschrieben und ergänzt werden oder in denen ein Großteil der Funktionalität außerhalb der Feld- und Methodendefinitionen stattfindet (zum Beispiel durch Nutzen von \texttt{init} oder \texttt{set!}), sind für die hier entstandene Implentierung von Mehrfachvererbung weniger gut geeignet. Eine Klasse, die in einer solchen Hierarchie durch Mehrfachvererbung entsteht, wird nur Zugriff auf einen sehr geringen Teil der Funktionalität ihrer Superklassen haben.

\section{Fazit}
Es ist mit Hilfe von Makros und dem Metaobjektprotokoll gelungen, einen Prototypen für Mehrfachvererbung in Racket zu implementieren, der die im Entwurf formulierten Anforderungen erfüllt.

Die Anforderung, dass die Erzeugung von Klassen und Objekten weiterhin von dem zugrundeligenden Objektsystem übernommen werden sollte, nimmt es uns ab, Klassen komplett neu definieren zu müssen. Stattdessen muss lediglich eine Superklasse erstellt werden, die die zu erbenden Eigenschaften vereinigt. Das Metaobjektprotokoll ließ sich leicht auf diesen Unterschied anpassen.

Dadurch, dass sich die neue Funktionalität gut in das bestehende Objektsystem einpassen soll, müssen beispielsweise Felder im Gegensatz zu CLOS manuell in der Klassendefinition von erbenden Klassen sichtbar gemacht werden. Einschränkung dieser Art hatten jedoch keinen großen Einfluss auf die generelle Implementierung des Metaobjektprotokolls.

Der Quelltext ist außerdem sehr kompakt; bis auf wenige Hilfsmethoden wurde er komplett im Kapitel Umsetzung abgedruckt.

CLOS hat im Gegensatz zu Racket jedoch nur eine sehr kleine Menge an Befehlen: es gibt Klassen mit Slots und Klassenoptionen, vier Arten von Methoden und generische Funktionen. Racket bietet etliche Klassenoptionen, sogar verschiedene Klassenmakros.

Für den Prototyp wurden diese Optionen auf eine Teilmenge beschränkt, die sehr ähnlich zu dem Umfang von CLOS ist: Felder und mit \texttt{define/public} definierte Methoden. Eine Erweiterung um andere (oder gar alle) Klassenoptionen würde sehr wahrscheinlich sowohl die Metaobjekte als auch das Bestimmen der zu vererbenden Eigenschaften deutlich komplexer machen. 

Die Implementierung folgt der in AMOP beschriebenen Implementierung von Closette, einem sehr vereinfachten CLOS-Interpreter, der die Verständnisgrundlage für die später im Buch vorgestellte, vollständige CLOS-Implementierung bildet. In den Worten der Autoren: ``Closette is simple, but it is woefully inefficient'' \cite[S.45]{amop}. 

Für kleine Klassenhierarchien wie das Pokemonbeispiel ist kein deutlicher Unterschied in der Übersetzungszeit messbar, aber für größere Programme kann das Vorgehen, das für den Prototyp verwendet wurde, zu einem Effizienzproblem werden. Die Grundidee der Optimierung von CLOS in AMOP ist Memoisierung: kritische Komponenten nur einmal zu berechnen und dann wiederzuverwenden. Es kann bei größeren Projekten oder mehr erlaubten Klassenoptionen notwendig sein, auch im Prototyp mit Memoisierung zu arbeiten.

Insgesamt lässt sich jedoch sagen: das Metaobjektprotokoll eignet sich zur Erweiterung eines bestehenden Objektsystems. Obwohl das zugrundeliegende Objektsystem von Racket sich deutlich von CLOS unterscheidet, waren nur sehr geringe Anpassungen des in AMOP vorgestellten Metaobjektprotokolls notwendig, um einen funktionierenden Prototyp zu erhalten.