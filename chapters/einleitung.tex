\chapter{Einleitung}
Funktionale Programmierung mit Racket ist Inhalt des Moduls ``Softwareentwicklung III'' an der Universität Hamburg. Ein Bestandteil des Moduls ist auch ein Einblick in objektorientierte Programmierung mit Racket. Die Studierenden sind bereits aus dem Modulen ``Softwareentwicklung I'' und ``Softwareentwicklung II'' mit Java vertraut. Das Objektsystem von Racket bietet damit kaum neue Lehrinhalte. Aus diesem Grund wird bisher Das Common Lisp Object System (CLOS), das in Racket als Paket zur Verfügung steht, behandelt. 

Ziel dieser Arbeit ist es, ausgewählte Konzepte aus CLOS auch zu dem integrierten Objektsystem von Racket hinzuzufügen. Der Fokus liegt dabei auf Lehrinhalten des Moduls ``Softwareentwicklung III''.

\section{Stand der Forschung} 
\begin{itemize}
 \item ganz knapp: Anfang von OOP (Warum hat man sich das mal überlegt?)
 \item welche Arten von OOP haben sich etabliert? Kernpunkte der Objektsysteme
 \item wichtig: Warum macht man das?
 \item von den Arten zu dem, was man sich ansehen will
 \item vllt. Paper zu CLOS
 \item NICHT Gott und die Welt zitieren! Nur relevantes!
\end{itemize}


\section{Vorgehen}
Zunächst wird die \textbf{Problemstellung} näher ausgeführt. Ein Kapitel zu \textbf{Racket} erläutert, wie die beiden Objektsysteme benutzt werden und beleuchtet die Implementation beider Ansätze. Anschließend wird anhand eines \textbf{Entwurf}s darauf eingegangen, auf welche Weise eine Umsetzung der CLOS-Konzepte im Objekt-System von Racket erfolgen kann und eine entprechende \textbf{Implementation} vorgestellt. Zum Abschluss wird im Kapitel \textbf{Zusammenfassung und Ausblick} über die Umsetzung reflektiert und Möglichkeiten der Erweiterung diskutiert.